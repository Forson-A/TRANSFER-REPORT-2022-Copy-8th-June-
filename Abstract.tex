%\begin{abstract}
\large{
    Understanding the structure and dynamics of our galaxy is complicated by the fact that our viewing point is positioned within the deferentially rotating disc of stars, gas, and dust clouds. Therefore, we get a different view of our galaxy that is unlike the view of an external galaxy. The studies at optical wavelength have limitations in probing the structure of our galaxy due to obscuration caused by dust clouds. 
In contrast, radio waves allow us to see through the dust. Hence use of the 21cm line emission (and absorption) provides a powerful tool to probe the galactic structure and dynamics  by measuring the distribution and velocities of neutral hydrogen clouds. Through several such and related studies, radio astronomers have made significant contributions towards revealing the 3D structure of our galaxy, including identification of the spiral structure.
%By investigating other galaxies' \cite{scannapieco2011formation},it has been revealed that spiral arms frequently follow a logarithmic spiral pattern. 


%Understanding the structure and dynamics of our galaxy is complicated by the fact that our viewing point is positioned within the rapidly rotating disc of stars, gas, and dust clouds that surrounds us and extends into space. In contrast to visible light, radio waves can pass through the dust clouds that obscures the visible light from stars, and by measuring the distances and velocities of neutral hydrogen clouds, radio astronomers have made significant contributions to  [reveal the 3D structure of our galaxy.] %the construction of a image of our spiral galaxy as seen perpendicular to the plane using radio astronomy telescopes.

%In order to conduct successful radio astronomy research, it is necessary to design and develop  [ suitable antenna system and receivers, which involve use of RF and digital electronics, and electrical as well as mechanical elements] %the multiple interacting radio frequency (RF), electrical, and electronics components, as well as the mechanical elements that are required to construct radio telescope. 

%These systems must be capable of detecting faint radio signals from space with desired resolution, accuracy and efficiency. Radio aperture synthesis (RAS) is a method developed by Sir Martin Ryle that involves electronically correlating the output of two or more aerials used as an interferometer \cite{graham1986martin}, in order to reconstruct an image which is equivalent in resolution to one observed from a large single telescope.\cite{lazio2019radio}. 

It was not until 1953 that astronomers realised the galaxy had a spiral structure, when the distances between star associations were precisely determined for the first time \cite{unsold2002structure}. 

%Because we are placed inside the deferentially rotating disc of stars, gas, and dust, we get a different view of our galaxy that is unlike the view of an external galaxy. The identification of the spiral structure using optical methods is highly challenging due to their ability to penetrate dust clouds that obscures visible light from stars.
%By investigating other galaxies' \cite{scannapieco2011formation},it has been revealed that spiral arms frequently follow a logarithmic spiral pattern. 


Neutral hydrogen (HI) makes up a significant portion of the interstellar medium (ISM), and since its discovery in 1951, its 21-cm line emission has been extensively studied at radio frequencies to gain a better understanding of galaxies, particularly the Milky Way galaxy \cite{storey199421}.
Specifically, using 21 cm surveys of  galactic HI emissions, it has been shown that it is possible to trace the ISM structure in part because the brightness temperature of the line is directly proportional to the column density of gas \cite{crovisier1983spatial}. 
Atomic hydrogen serves as a connection between the gas heated or ejected by big stars and the cold molecular gas from which new stars are formed. HI is widely dispersed and, within certain limitations, is easily observable across the galaxy through the 21 cm line.



Radial velocities calculated from H1 spectra have been used to map the locations of gas clouds in the galaxy's plane \cite{hossain2018salsa}, with the goal of determining the galaxy's essential structure through the mapping of gas clouds. The disk structure of the Milky Way galaxy and its spiral arms were discovered using the early single-dish antenna scans according to \cite{dickey2001southern,mcclure2001southern}.
By utilising interferometry techniques, the structure of HI in the Galactic plane may be seen over a wide range of angular scales, down to an arc-minute in emission or less than an arc-minute in absorption  \cite{deshpande2000power,crovisier1985observation,kalberla1985high} . But the structure of H1 in the galactic plane cannot be easily described in terms of discrete clouds \cite{deshpande2000power, clark1965interferometer, clark1962hydrogen}, with the exception of a few gas clouds associated with supernova remnants or HII regions \cite {green1993power, sturner1994association, routledge1991structure}. 
%..................................
%The angular power spectrum of H1 emission across a range of scales may be depicted in terms of its angular power spectrum rather than single dish-antenna surveys that reflect consistent scales.
%....................................

The angular power spectrum can be estimated using an array of telescopes with different and shorter baselines, which can be used to determine the structures present at various scales and understand the underlying processes. However, only a limited amount of work has been done in this area so far \cite{green1993power}, with results that are somehow confusing when compared to single-antenna observation results. From observations using the Nancay and Arecibo telescope, \cite{dickey2001southern} derived the power spectrum as \(\propto {a^-}{^2}\), and from observations with synthesis
telescope of the Dominion Radio Astrophysical Observatory (DRAO) by \cite{green1993power} the spectrum was seen to be \(\propto {a^-}{^2}{^.}{^2}\),
which are quite different from the \(\propto {a^-}{^3}\) trend reported from Westerbork Synthesis Radio Telescope (WSRT) measurement of  presented by \cite{crovisier1983spatial}. where: a is the spatial frequency.

In order to probe the large-scale structure of the H1 emission, shorter baselines are essential, and have indeed been  used to advantage in the previous studies by \cite{green1993power} and \cite{dickey2001southern} utilising interferometry techniques. However, the size of their dishes constrained their ability to explore the shorter baselines.
%.........................................%This is related to the fact that shorter baseline spectra are highly associated with one another, and that shorter baseline spectra correspond to large angular scales.
%.....................................
%Shorter baselines have been shown to be advantageous for analysing the structure of the H1 emission in previous research by 




 Advancements in the study of atomic HI in the universe have been identified as one of the primary scientific drivers for the development of the global Square Kilometre Array (SKA) project \cite{taylor2012square, schilizzi2011project} which would be the world's largest radio telescope, as described by \cite{taylor2012square} and others  \cite{wang2020ska}. SKA1-MID is a dish-based component that is currently being built in South Africa. The second phase of the component, known as SKA2-MID, will be extended to eight other African partner countries by 2030, including Ghana. Kenya, Zambia, Namibia, Botswana, Madagascar, Mozambique, and Mauritius according to \cite{hoare2018uk}.


The goal of our proposed study is to design and build a small array of parabolic dish antennas with shorter baselines that will be capable of analysing the angular scale of galactic HI emission from the southern hemisphere and will also serve as a precursor instrument for the SKA2-MID in Mauritius when it is completed, among other things. 

In order to conduct successful radio astronomy research, it is necessary to design and develop suitable antenna and receivers system, which involve use of radio frequency and digital electronics, and electrical as well as mechanical elements.
These systems must be capable of detecting faint radio signals from space with desired resolution, accuracy and efficiency. %Radio aperture synthesis is a method developed by Sir Martin Ryle that involves electronically correlating the output of two or more aerials used as an interferometer \cite{graham1986martin}, in order to reconstruct an image which is equivalent in resolution to one observed from a large single telescope.\cite{lazio2019radio}. 

This entails a subjective technical examination of the subject, as well as a complete report on the scientific needs of the system. Key results of the comprehensive mechanical design on the dish conversion and construction narrative of the tracking system of parabolic dish antenna, as well as the signal processing receiver and data pipeline, with detailed reporting on measurements and analytical estimation.

At the Mauritius Radio Telescope observatory site, four 2.4-meter redundant telecommunication satellite offset parabolic dishes are being converted into radio astronomy telescopes for the development of an array of parabolic dish antennas.
%......................................
%There has been a We have  successfully completed the analysis and optimisation of the Mauritius cyclonic condition as well as other atmospheric properties that are required for designing and engineering of mechanical structures, specifically for radio astronomy and antenna engineering.The mechanical construction sections of the antenna were successfully designed, examined, and simulated before being constructed and assembled. 
 We exploited the Mauritius cyclonic weather conditions and other atmospheric factors in designing and building the antenna mechanical structures.
%This made it easier to study and improve mechanical structures, which is important for radio astronomy and antenna engineering.
Before construction and assembly, the antenna system's mechanical construction parts, antenna tracking system, feed horn design, and signal and electrical cabling systems were successfully investigated, designed, and simulated.
Our study into the design, optimisation, and analysis of the array configuration and commissioning will be employed in the future work to ensure that the telescopes performs  the required scientific goals.
 

%We have also done study in configuring four such dishes in an array. The design and configuration we have arrived at is being used in the construction of the array.  

%These instrument will be capable of measuring the angular power spectrum of the HI emissions using interferometric radio observations, sampling a variety of baselines, and measuring the closure phase of the HI emission when the array is successfully commissioned.


When the array is commissioned and fully operational, the instrument will be capable of measuring the angular power spectrum of the HI emissions by interferometric radio observations, as well as sampling a wide range of baselines and measuring the HI emissions' closure phase.

%The instruments, which include Mauritius' first parabolic dish antenna for radio astronomy research, will enable the development of the key skills and capacity to support the SKA2-MID outstations to be constructed in Mauritius. As well as the African Very Long Baseline Interferometry Network (AVN) project aim.

The instruments, which include Mauritius' first parabolic dish antenna for radio astronomy research, will support the development of skills and build institutional capacity on the continent and at the SKA2-MID outstations to be built in Mauritius. In addition to the African Very Long Baseline Interferometry Network (AVN) project aims.
\cite{copley2016african} %and \cite{32mkuntuantenna}
    
    
}
    
    
%\end{abstract}