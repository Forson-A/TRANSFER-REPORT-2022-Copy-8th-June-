\chapter{Literature Review}
\label{Chapter 2}

%\section{Literature Review}

\large {
\section*{Background}
This chapter provides a brief introduction to radio astronomy, instrumentation for
radio astronomy, and instrumentation for HI science in particular. The work presented in this thesis made extensive use of the design , simulation software and building of hardware tools.
and we provide an overview of the HI power spectrum requirements  and approach in this chapter




\textbf{Background}\\
Development of an arrays for research in the radio astronomy field necessitates  the engineering designs of the various associated components and parts that will enable fast and accurate detection of the radio signals coming from space. In order to perform this task, an understanding of the instrument
and how it functions and develop is necessary. In this sections a brief
discussion is given on the instrument's fundamentals components and how the signals from space are  detected.


\section{Fundamentals of Radio Astronomy Instrument Design}

Dated back in 1932 \cite{jansky1933radio} 
when Karl Jansky discovered the first radio waves emanating from the Milky Way, the history of the radio astronomy has been one of leading in solving many of the engineering problems for constructing radio telescopes and  continually increasing angular resolution.
Designing and development of the 21st century radio telescopes, requires orders in magnitude more sensitive than the present day radio telescopes and exceptional immunity to man-made interference to operations. %\cite{smits1998square}.
Due to these interference, sensitive radio astronomy instrumentation to observe the radio sky at millimetre and sub-millimetre wavelengths and high resolutions emerged.
Designing of radio antennas and their associated components today enjoys the convenience of the instruments in performing fast, accurate, automated, design-related calculations due to the availability of a multitude of software “design tools'' for modelling electromagnetic fields.
 One of the most important components of a radio telescope system is the antenna. Many systems use directional antennas, which must therefore be pointed in a desired direction \cite{smith2012antenna}. 
 In order to maintain accurate pointing, the antenna requires a stable mount in a fixed location. This antenna mounting is the mechanism deployed to provide rotational and directional pointing of antenna assembly. This is because wind load could cause the deformation of the reflector surface, which in turn could seriously affect the resolution and the sensitivity of the antenna and consequently degrade its performance. \%cite{liu2018seismic}
 pointed out that the analysis of the wind characteristics and the science goals of the antenna mount is particularly important.It is this background, that requires investigation on the wind load analysis towards the development of a suitable mount, feed, dish tracking system and array configuration for an array of small parabolic antennas at the MRT site for radio astronomy experiments.The development of prototype instruments for radio astronomy forms a major part of astronomy research
 
Development of prototype telescopes  are like bread and butter for astronomers as they form the starting point of identification of path finding instrumentation be developed with well-defined criteria. A set of ingenious techniques have been developed over the past 50 years to construct and use synthesis aperture arrays \cite{mcmahon2011adventures}
is the key idea behind these class of telescopes.
The invention and use of some of the core ideas which is the interferometry-based radio astronomy (which are often referred to as radio interferometers, phased arrays, or radio telescope arrays), makes it possible to somehow combine the signals from a set of several single-dish antennas to obtain a new signal that results in an observation with far better angular resolution than the individual dishes are capable of. More precisely, if the distance between the two most separate dishes in the set is l, then it is possible to synthesise an aperture with an effective diameter of D = l.
 Radio astronomy is the weakest wavelength limited by ionosphere which reflects its waves when it passes through the atmosphere to the surface of the Earth.









\section{History of Radio Astronomy }

Section stuff

\section{ An Engineer’s View of Radio Astronomy and Instrumentation}
\section{Single Antenna Radio Astronomy}
Section stuff

\section{Reflectarray Antennas
}


literature from this book bolow....

file:///D:/UoM/e-books/Books%20on%20Antennas/Reflectarray%20Antennas%20Theory,%20Designs%20and%20Applications%20by%20Payam%20Nayeri,%20Fan%20Yang,%20Atef.%20Z.%20Elsherbeni%20(z-lib.org).pdf


\section{Radio Telescope Arrays}
Section stuff



\section{Neutral Hydrogen Emissions (HI)}

Neutral hydrogen (HI) makes up a significant portion of the interstellar medium (ISM), and since its discovery in 1951, its 21-cm line emission has been extensively studied at radio frequencies to gain a better understanding of galaxies, particularly the Milky Way galaxy \cite{storey199421}.
Specifically, using 21 cm surveys of  galactic HI emissions, it has been shown that it is possible to trace the ISM structure in part because the brightness temperature of the line is directly proportional to the column density of gas \cite{crovisier1983spatial}. 
Atomic hydrogen serves as a connection between the gas heated or ejected by big stars and the cold molecular gas from which new stars are formed. HI is widely dispersed and, within certain limitations, is easily observable across the galaxy through the 21 cm line.

Radial velocities calculated from HI spectra have been used to map the locations of gas clouds in the galaxy's plane \cite{bekhti2016hi4pi}, with the goal of determining the galaxy's essential structure through the mapping of gas clouds. The disk structure of the Milky Way galaxy and its spiral arms were discovered using the early single-dish antenna scans according to \cite{dickey2001southern,mcclure2001southern}.
By utilising interferometry techniques, the structure of HI in the Galactic plane may be seen over a wide range of angular scales, down to an arc-minute in emission or less than an arc-minute in absorption  \cite{deshpande2000power,crovisier1985observation,kalberla1985high} . But the structure of H1 in the galactic plane cannot be easily described in terms of discrete clouds \cite{deshpande2000power, clark1965interferometer, clark1962hydrogen}, with the exception of a few gas clouds associated with supernova remnants or HII regions \cite {green1993power, sturner1994association, routledge1991structure}. 


Advancements in the study of atomic HI in the universe have been identified as one of the primary scientific drivers for the development of the global Square Kilometre Array (SKA) project \cite{taylor2012square, schilizzi2011project} which would be the world's largest radio telescope, as described by \cite{taylor2012square} and others  \cite{wang2020ska}. SKA1-MID is a dish-based component that is currently being built in South Africa. The second phase of the component, known as SKA2-MID, will be extended to eight other African partner countries by 2030, including Ghana, Kenya, Zambia, Namibia, Botswana, Madagascar, Mozambique, and Mauritius according to \cite{hoare2018uk}.






\section{The angular power spectrum of Galactic Hydrogen Emission HI}



\section{The angular power spectrum of HI}









%NOTE THE LACK OF A BIBLIOGRAPHY CALL IN THIS FILE. BIBLIOGRAPHY WORK HAPPENS OUTSIDE THE CHAPTER TEX FILES.
















}











